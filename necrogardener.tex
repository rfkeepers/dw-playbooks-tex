\documentclass[darkmode]{dw_playbook}
% bright white computer screen hurting your eyes?  Specify [darkmode] to invert the color scheme.

%%%%%%%%%%%%%%%%%%%%%%%%%%%%%%%%%%%%%%%%%%%%%%%%%%%%%%%%%%%%%%%%%%%%%%%%%%%%%%%%
%    Playbook Details
%%%%%%%%%%%%%%%%%%%%%%%%%%%%%%%%%%%%%%%%%%%%%%%%%%%%%%%%%%%%%%%%%%%%%%%%%%%%%%%%

\setPlaybook{Necrogardener}
\title{The \playbook}

\setDamage{\dsix}
\setMinHitPoints{6}

%%%%%%%%%%%%%%%%%%%%%%%%%%%%%%%%%%%%%%%%%%%%%%%%%%%%%%%%%%%%%%%%%%%%%%%%%%%%%%%%
%    Page Layout
%%%%%%%%%%%%%%%%%%%%%%%%%%%%%%%%%%%%%%%%%%%%%%%%%%%%%%%%%%%%%%%%%%%%%%%%%%%%%%%%

\begin{document}
% ----------------------------------------------------------
\pageOne
    % any text that appears betweeen the top banner and the stats
    {
        \names{Hasil, Patricia, Hank, Kimberly, Iggy, Jimbo, Chuck, Bettie, Peg, Sarah, Darlene, Ethel, Walter, Earl, Leonard, Oswald}
    }
    % left-hand column (looks, race)
    {
        \looks
        \looksOption{EYES}{Blood red, Vivid green, or Yellow puss}
        \looksOption{HAIR}{Waxy, Stuffed under a wide-brimmed hat, or Falling out}
        \looksOption{BREATH}{Bitter cold, Hot and humid, or Like fresh compost}
        \looksOption{BODY}{Smooth and pallid, Plump, or Mossy and possibly disintegrating}

        \vspace{1mm}
        \race\\
        \raceOption{Human}
            {When you \important{return to a civilized settlement where you used to garden}, tell the GM about the garden as you left it: what plants were inside?  Were they for produce, beauty, or a more twisted growth?  The GM will tell you what has escaped from the garden since that time.}
        \raceOption{Orc}
            {When you spend a day \important{growing vines into the skeleton of a beast}, you raise a shambling, mindless thrall capable of hauling heavy loads.  The thing is lifeless: it needs no water, food, nor rest, and will plod along when prodded or whipped.}
        \raceOption{Elf}
            {When you \important{loose your beetles and maggots into a recently dead creature} they consume the body and leave pure, black soil in their wake.  Larger creatures take longer for your bugs to break down, but create more soil as a result.}
    }
    % Starting Moves
    {
        \move[x]{Grow Golem}
            {While you have a bit of downtime (\important{Make Camp}, \important{Recover}, \important{Bolster}, or other sort of idle time), you may plant a tooth, beak, claw, or bone in some soil to see what grows.  In six hours your seed will produce a small, grotesque plant bearing a single engorged fruit.  Peeling back the fruit’s flesh will birth a tiny golem made from bone, roots, clay, and fungus, who follows you around like a faithful pet.
            \gap
            Your golem is about a foot tall, can carry 3 load, deals D4 damage, and has 6 maximum HP.  It is sentient and will obey any command you give it.  Only one golem is ever functional at a time.}
        \gap
        \move[x]{Harvest}
            {When you \important{drain the life from your golem’s body}, reach out and make physical contact with it, then gain HP equal to its remaining HP.  This kills the golem.}
        \gap
        \move[x]{Rot Gourmand}
            {Eating spoiled, rotting, or pestilent food is the same as any other meal to you: just as filling and no less healthy.  When you are in a place where food waste can be scavenged, you are not required to consume a ration when a move would have you do so.
            \gap
            When you \important{touch food or drink with your bare skin} you may cause it to spoil, sour, foul, or bitter.  If the soiled meal would normally slake, satisfy, refresh, or heal, it loses those benefits for anyone but yourself.}
        \move[x]{Psychotropics}
            {When you take a deep breath and \important{exhale the spores from your moldy lungs} into someone’s face, whomever breathes your exhalation finds themselves under its effects: name an emotional state you want the inhaler to experience and roll+CHA.  On a 7-9 the GM picks one of the following side-effects.  On a miss the inhaler goes into a state, but not the one you anticipated.
            \gapSm
            \opt{Side effects may include: Inability to feel pleasure, Drowsiness, Impaired Coordination, Asthma and Difficulty Breathing, and Insatiable Appetite}}
        \gap
        \move[x]{Blacksoil}
            {When you \important{sow a seed in pure, black soil} to grow a plant or creature, tell the GM what abomination you’re trying to produce.  Ritual growth is always possible, but the GM will give you one to four of the following conditions:
            \gapSm
            \opt{It’s going to take days/weeks/months.}
            \gapSm
            \opt{The best you can do is a lesser version, unreliable and weak.}
            \gapSm
            \opt{You and your allies will risk danger from \blank{3}.}
            \gapSm
            \opt{First you must enrich the soil by fertilizing it with \blank{3}.}
            \gapSm
            \opt{The growth will drain away all \blank{3} nearby.}
            \gapSm
            \opt{It’ll require a lot of resources.}}
    }

% ----------------------------------------------------------
\clearpage
~

\pageTwo
    % left-hand column (bonds, alignment, gear)
    {
        \bonds
        \bondOption{\blankName~has tasted the fruits I grow and enjoys their flavor.}\\
        \bondOption{\blankName~thinks what I grow are perversions, but they couldn’t grow a weed if someone planted it for them.}\\
        \bondOption{I want a particularly rare plant, and \blankName~is the perfect guide to help me find it.}\\
        \bondOption{\blankName~once came to me to buy components for a poison.}\\
        
        \vspace{2mm}
        \blank{5.8}\\
            
        \blank{5.8}\\
            
        \blank{5.8}

        \vspace{1mm}
        \alignment
        \alignmentOption{Lawful}{Make a plan and stick to it.  Weed out any unexpected addition, independent of its benefit.}
        \gap
        \alignmentOption{Chaotic}{Permit an unsavory creature to live as long as it doesn’t intentionally harm you.}\gapSm
        \diyMove{5}

        \gear
        \adventuringGear
        \gearOption{\important{Bag of Pure, Black Soil} (1 use, 1 weight)}
        \gearOption{An iron \important{Ritual Trowel} (\tag{hand}, 1 weight)}
        \gap
        \instruction{Choose your Armament:}
        \gapSm
        \gearOptionSelectable{\important{Sickle} (\tag{close}, 1 weight)}
        \gearOptionSelectable{\important{Sharpened Shovel} (\tag{close}, \tag{two-handed}, 1 weight)}
        \gap
        \instruction{Choose your Clothing:}
        \gapSm
        \gearOptionSelectable{\important{Decaying Leather} (0 armor, \tag{worn}, 1 weight)}
        \gearOptionSelectable{\important{Sturdy Cottons} (0 armor, \tag{worn}, 1 weight)}
        \gap
        \instruction{Choose one:}
        \gap
        \gearOptionSelectable{\important{6 coins}}
        \gearOptionSelectable{\important{Maggot-ridden Hardtack} (\tag{ration}, \tag{spoiled}, 5 uses, 1 weight)}
    }
    % Advanced Moves (2-10)
    {
        \move{Bloodmeal}
            {When you \important{work blood, entrails, and bone into the soil} around the roots of a mature plant, it explodes in growths of rich and fleshy fruits positively bursting with raw meat.  Within an hour the plant will produce enough to heartily fill the stomachs of six, then over the course of the next day the entire plant will rot and die.}
        \gap
        \move{Blight}
            {When you \important{spread a malignant rumor to corrupt a belief}, roll+CHA.  On a 10+, choose 2.  On a 7-9, choose 1.  On a miss, the idea is introduced, but the results are far from what you wanted.
            \gapSm
            \opt{The rumor is infectious.  A single person will quickly come to agree with your idea.  In a group, the idea swiftly spreads between all members.}
            \gapSm
            \opt{The rumor is obstinate.  The person or people who hear it are consumed with the knowledge, unable to remove it from their thoughts.}
            \gapSm
            \opt{The rumor is subtle. Those exposed to it will not take extreme measures to cure themselves of the knowledte, or remove its source.}}
        \gap
        \move{Edibles}
            {When you \important{inoculate a fresh corpse with spores}, they swiftly spread into the warm meat and, in a matter of seconds, produce a gooey, orange ball of fungus that can be safely plucked and eaten.  Whomever consumes the ball heals for 1d4+CON (their CON).}
        \gap
        \move{Reaper}
            {When you or your golem \important{deal the killing blow}, you both gain health equal to 1d4+CON.}
        \gap
        \move{Corpse Bomb}
            {When you \important{command your golem to self-destruct} it explodes in a blast of wooden, bony shrapnel.  Everyone within \tag{reach} of the golem takes damage equal to its remaining HP.}
        \gap
        \move{Putrefying Toxin}
            {When any creature but yourself \important{consumes a meal you’ve spoiled}, the meal cannot slake, satisfy, refresh, or heal, and instead you deal damage twice against that creature.  Anyone who survives this meal becomes severely ill and finds it painful and exhausting to move.}
        \gap
        \move{Work Smarter, Not Harder}
            {In a garden, every living thing dies a different way.  You need to know what you're killing to understand how to kill it.  When you \important{Hack and Slash} with a gardening tool, roll+INT instead of +STR.}
        \gap
        \move{Fly, My Pretties}
            {Your golems have a set of tattered wings which they can use to fly.  Their capacity for flight is more bumbling than agile... but it gets them where they’re going.}
        \gap
        \move{Multiclass Dabbler}
            {Get one move from another class. Treat your level as one lower for choosing the move.}
    }
    % Advanced Moves (6-10)
    {
        \move{Pinch Back The Growth}
            {With a little daily care and attention you can extend the growth of your golem plant.  Every day you \important{tend to the same plant} it grows larger, as does the golem it produces.  The plant reaches its full potential somewhere between 4 to 7 days, being about the height of an apple tree and bearing a fruit the size of a small bear.
            \gap
            As the plant grows larger so do your golem’s stats, increasing up to: 10 load, D10 damage, and 16 maximum HP.  If you use Corpse Bomb on one of these larger golems, the explosion is \tag{forceful}.}
        \move{Pesticide}
            {When you \important{Hack and Slash} with a creature that is more of a pest or nuisance than a formidable foe, skip the damage roll and just kill the thing.}
        \gap
        \move{Side Effects May Vary}
            {When you \important{exhale psychotropic spores}, instead of triggering an emotion you may pick one of the listed side effects.  On a hit, the onset of that side effect is sudden and severe.  On a 7-9 the GM adds an additional side effect or an emotion.  On a miss the inhaler experiences some side effect, but not the one you wanted, and perhaps not even one you’ve ever caused before.}
    }

% ----------------------------------------------------------
\clearpage
~

% the third page has empty lines for notes on the left-hand column, and I don't
% leave you any options for changing that up.  Everything you can configure here
% goes into the right hand column.
\pageThree
    % full-width section for an optional title bar.
    {
        \advancedMovesCont
    }
    % two-column section for listing moves
    {
        \move{Forbidden Fruit}
            {When you use blacksoil to \important{grow a delicious fruit that is plump with knowledge}, you may eat the fruit to gain the forbidden knowledge, or you may try to convince another character to consume it.  But remember, some knowledge is better avoided, for your own sake.
            \gap
            If a PC eats the fruit, they may ask the GM a single question about any topic- knowable or unknowable- and the GM will answer truthfully.  Afterward, the GM will reveal or introduce a second truth to the player, something damning and unrecoverable.
            \gap
            If you convince an NPC to eat the fruit, you then reveal a truth to them.  They are unable to deny this truth, unable to rationalize it, and unable to forget it, regardless of how shameful or upsetting the knowledge may be.}
        \gap
        \move{Thorny}
            {\instruction{You cannot take NOT THE BEES!.}
            \gapSm
            Your skin permanently grows thorny extrusions.  Describe what plant these thorns most closely resemble: brambles, cactus, locust trees, or something else?  You have +2 ongoing for any roll where you resist a creature who tries to grapple or physically move you against your will.
            \gap
            Choose a poison.  Your thorns exude this poison, and you are now immune to its effects.  Ignore the poison's original tags and give it the tag \tag{stabbed}:  The thorns are not \tag{dangerous} for anyone who merely brushes up against you.  It takes a good, strong stab that breaks skin to apply the effects.}
        \move{NOT THE BEES!}
            {\instruction{You cannot take Thorny.}
            \gapSm
            Your body hosts a hive of flying, stinging, biting insects. When you \important{expose the entrance to your hive} you release a swarm into the air.  The insects implicitly understand your wants, and will go and do whatever you think they should.  The size of the swarm is large enough to attack a specific creature, or can spread out to frustrate everyone in a room or small space.  Only the mature, militant members leave the hive, while the breeders and young remain inside.  If the swarm is decimated, it doesn’t take very long to repopulate.
            \gap
            When you \important{send your swarm of insects to attack}, roll+WIS.  On a 10+ the critters do their work: deal your damage.  On a 7-9, deal your damage and choose one:
            \gapSm
            \opt{The target does a decent job keeping your insects at bay: -1d4 damage.}
            \gapSm
            \opt{Much of the swarm is killed in the process, you may not attack with this move until it repopulates.}
            \gapSm
            \opt{The swarm takes time to exit the hive, leaving you open to danger.}}
        \gap
        \move{Multiclass Initiate}
            {Get one move from another class. Treat your level as one lower for choosing the move.}
        \gap
        \move{Multiclass Master}
            {Get one move from another class. Treat your level as one lower for choosing the move.}
    }
    % and/or a single-column section if you want to use the third page for
    % something with a different layout than before
    {
        ~
    }

% ----------------------------------------------------------
% Hey there fellow playbook creator!  Thanks for trying out this latex project.  Formatting documents is a tough gig, and even with the tools arranged here I'm sure you'll go through a few headaches.  Feel free to reach out if you've hit a wall.

% I'm only going to ask you for one thing: please keep this final page intact.  Help other players and designers find this resource as well.  While you're at it, if you're making your own playbook you probably have a drivethru or itch.io account, if not your own website, right?  Add a link to it here!

% Thanks again, and happy designing!

\clearpage
~

\resourceLinks
    {
        % link to your stuff here!
        % ex:
        % More awesome playbooks can be found at at:\\\url{https://itch.io/myOtherCreations}
    }
\end{document}